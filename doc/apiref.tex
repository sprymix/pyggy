\chapter{API Reference}

This section describes the public interfaces provided by
the PyGgy package.

\section{\module{pyggy}}
\declaremodule{extension}{pyggy}
\modulesynopsis{The PyGgy and PyLly parsing and lexing package}
\moduleauthor{Tim Newsham}{newsham@lava.net}

The \module{pyggy} module is the container for the entire PyGgy
and PyLly package.  Importing this module loads in the ``simple''
API.   This API consists of a number of utility functions and
exception classes.

\begin{funcdesc}{pyggy.generate}{fname, targ, debug=0, forcgen=0}
This function takes in filenames specifying a specification file
and an output file name.  If the output file does not exist or
is older than the specification file, it processes the specification
file and generates tables into the target file.  The input specification
must be a \file{.pyl} or \file{.pyg} file.  If the \code{debug} argument
is specified, increased debugging information is emitted while
processing the spec file.  For
a description of the debug levels see the documentation on the
\module{pyggy.pyggy} and \module{pyggy.pylly} modules.  If \code{forcegen}
is true, the specification file is processed whether or not it is
newer than the target file.

If an invalid specification file is specified, \class{pyggy.ApiError}
is raised.  Any exception raised by \method{pyggy.pyggy.parsespec} or
\method{pyggy.pylly.parsespec} may also be raised.
\end{funcdesc}

\begin{funcdesc}{pyggy.getlexer}{specfname, debug=0, forcegen=0}
This function generates a lexer table module from a lexer spec file, imports
the module and returns a \class{pyggy.lexer.lexer} instance and the
module.  The arguments \code{debug} and \code{forcegen} have the
same meaning as in the \function{generate} function.

This function can raise any of the exceptions raised by \function{generate}.
\end{funcdesc}

\begin{funcdesc}{pyggy.getparser}{specfname, debug=0, forcegen=0}
This function generates a parser table module from a parser spec file,
imports the module and returns a \class{pyggy.glr.GLR} instance
and the module.  The arguments \code{debug} and \code{forcegen} have the
same meaning as in the \function{generate} function.

This function can raise any of the exceptions raised by \function{generate}.
\end{funcdesc}


\begin{funcdesc}{pyggy.proctree}{t, gram, allowambig=0}
This function post-processes a parse tree previously returned
from a call to a \method{parse} method.  It takes as arguments
the parse tree and the module containing the parse tables.  The
optional argument \code{allowambig} is used to specify that ambiguous
parses are allowed, otherwise a \class{pyggy.AmbigParseError} is
raised if any ambiguities are encountered in the parse tree.

The function walks the parse tree in a bottom-up fashion executing
the semantic action code for each production used in the derivation.
For each action executed, the list of the values from the right
hand side of the production are passed in.  These values are either
from the \code{value} fields of tokens, or the values previously
returned by other action code functions.  If ambiguous parses are
disallowed, each right hand side element is represented by the
element's unique value.  In the case that ambiguous parses are allowed
and an ambiguity is found in the parse tree, the ambiguous right
hand side element will be represented with a \class{pyggy.glr.symnode}
instance whose \code{possibilities} field is a list of the alternate
values. 

The \function{proctree} function returns the value associated with
the start symbol which is at the root of the parse tree.

This function may raise \class{pyggy.AmbigParseError} if an ambiguous
parse is detected or \class{pyggy.ApiError} if an invalid tree is passed in.
This function alters the parse tree as it operates on it.
\end{funcdesc}

\begin{excdesc}{pyggy.Error}
A superclass of all PyGgy generated exceptions.
\end{excdesc}

\begin{excdesc}{pyggy.InternalError}
An error in the inner workings of PyGgy or PyLly.  This is a type
of \class{pyggy.Error}.
\end{excdesc}

\begin{excdesc}{pyggy.ApiError}
This error specifies that there is an error in the way that
PyGgy or PyLly is being used.  This is a type of \class{pyggy.Error}.
\end{excdesc}

\begin{excdesc}{pyggy.LexError}
This error specifies that there was an error while lexing an
input source.  This exception is not currently used since the lexer
currently returns an error token on errors.  This exception is a
type of \class{pyggy.Error}.
\end{excdesc}

\begin{excclassdesc}{pyggy.ParseError}{str, tok}
This error specifies that an error occured during the parsing of
a token stream.  It stores a description of the error in \code{str}
and the token that caused the error in \code{tok}.  This exception
is a type of \class{pyggy.Error}.
\end{excclassdesc}

\begin{excdesc}{pyggy.AmbigParseError}
This error specifies that an ambiguity was present in a parse tree
when ambiguities are disallowed.  It is a type of \class{pyggy.ApiError}.
\end{excdesc}


\section{\module{pyggy.lexer} -- The lexing engine}
\declaremodule{extension}{pyggy.lexer}
\modulesynopsis{The PyLly lexing engine}
\moduleauthor{Tim Newsham}{newsham@lava.net}

\begin{datadesc}{pyggy.lexer.TOK\_ERR}
\code{TOK\_ERR} is returned by the lexer whenever a character is encountered
which cannot be lexed.
\end{datadesc}

\begin{classdesc}{pyggy.lexer.lexer}{lexspec}
The lexer class provides an extensible lexer class capable of lexing
tokens from an input source based on tables generated by PyLly.
The \code{lexspec} argument is passed in from a generated table module.
It should be a tuple made up of a DFA table, a list of start states,
a list of actions for each accepting state, a list of actions to
be performed at the end of file and a dictionary of character classes.
The format of the \code{lexspec} argument is subject to change.

The \class{lexer} class provides lexing from input strings or from
files.  The class can be subclassed to provide lexers with different
input behaviors.

\begin{memberdesc}{value}
The \code{value} member holds the value associated with the most
recently returned token.
\end{memberdesc}

\begin{methoddesc}{setinputstr}{str}
Sets the input source to be the characters in the \code{str} argument.
\end{methoddesc}

\begin{methoddesc}{setinput}{fname}
Sets the input to source to be the characters from the specified file.
A filename of \file{-} indicates data should come from \code{stdin}.
\end{methoddesc}

\begin{methoddesc}{token}{}
Returns the next token lexed from the input stream.  If there are no
more tokens left, the value \code{None} is returned.  If an error occurs
the value \code{pyggy.lexer.TOK\_EOF} is returned.  In the future an
exception may be raised in this situation.  Before a token is returned,
the \code{value} member is set to the value specified in the action
code for the token, or to a string of the characters in the token.
\end{methoddesc}

\end{classdesc}



\section{\module{pyggy.srgram}}
\declaremodule{extension}{pyggy.srgram}
\modulesynopsis{Shift-Reduce table handler}
\moduleauthor{Tim Newsham}{newsham@lava.net}

\begin{classdesc}{pyggy.srgram.SRGram}{srspec}
This class encapsulates the tables of a shift-reduce parser to
isolate the parsing engine from the details of the table implementation.
It's constructor takes a single argument which should be the
grammar table specification from a generated grammar table module.
The specification is a tuple of a GOTO table, an ACTION table and
a list of semantic actions.
The format of this specification is subject to change.
\end{classdesc}



\section{\module{pyggy.glr } -- The PyGgy parsing engine}
\declaremodule{extension}{pyggy.glr}
\modulesynopsis{The PyGgy parsing engine}
\moduleauthor{Tim Newsham}{newsham@lava.net}

This module implements the Generalized-LR parsing engine and provides
the supporting data structures and helper functions.


Parse trees are composed of alternating levels of \class{symnode}
and \class{rulenode} instances.  The root of the parse tree is
a \class{symnode} for the start symbol of the grammar.

\begin{classdesc}{pyggy.glr.symnode}{sym, possibilities, cover}
The \class{symnode} class is used to represent terminals and non-terminals
in the parse tree while parsing.  It's \code{sym} field is either
a tuple of the token name and value for terminals or a non-terminal
symbol.  The \code{possibilities} field holds a list of all possible
derivations of the current symbol.  Each element in \code{possibilities}
is a \class{rulenode} instance.
\end{classdesc}

\begin{classdesc}{pyggy.glr.rulenode}{rule, elements, cover}
The \class{rulenode} class is used to represent a production used
in the derivation.  It's \code{rule} member specifies which production
in the grammar the \class{rulenode} represents.  It is a tuple
of the left hand side symbol, a count of right hand side elements and
the production number in the grammar.  The \code{elements} member
is a list of \class{symnode}s for the right hand side of the
production.
\end{classdesc}

\begin{funcdesc}{pyggy.glr.dottree}{tree, printcover=0}
This is a helper for visualizing a parse tree.  It uses the
\file{dotty} program to show a graphical representation of the
parse tree \code{tree}.  If \code{printcover} is true, the
token positions each tree node covers is also displayed.
\end{funcdesc}

\begin{classdesc}{pyggy.glr.GLR}{gram}
The \class{GLR} class implements the GLR parsing engine.  It takes
in a single argument which is a \class{pyggy.srgram.SRGram} instance.

\begin{methoddesc}{setlexer}{lex}
Sets input to come from the lexer \code{lex}.  This must be called
before parsing is started.
\end{methoddesc}

\begin{methoddesc}{parse}{}
Parses the stream of tokens from the designated lexer and return
a parse tree.

This method may raise \class{pyggy.PaseError} or \class{pyggy.InternalError}.
\end{methoddesc}
\end{classdesc}



\section{\module{pyggy.pylly} -- Lexer generation}
\declaremodule{extension}{pyggy.pylly}
\modulesynopsis{Lexer generator}
\moduleauthor{Tim Newsham}{newsham@lava.net}

This module implements the lexer generator.  It is responsible for
reading in a lexer spec file (in \file{.pyl} format), generating
finite state machines and emitting tables for the machines into a
Python module.

This module can be accessed from the command line or through a call
from Python.  To run from the command line:

\begin{verbatim}
$ python pylly.py [-d debuglevel] infile.pyl outfile.py
\end{verbatim}

\begin{funcdesc}{pyggy.pylly.parsespec}{fname, outfname, debug=0}
The \function{parsespec} function causes the input file \code{fname}
to be processed and finite state machine tables to be generated to
\code{fname}.  If \code{debug} is set, increasing amounts of diagnostic
output will be emitted.  The debug levels (especially higher-levels)
are subject to change but are currently:

\begin{tableii}{l|l}{exception}{Level}{Description}
  \lineii{0} {Output a count of ambiguities in the lexer.}
  \lineii{1} {Output detailed diagnostics of the generated lexer.}
  \lineii{2} {Show the DFAs constructed for each start state with \file{dotty}.}
  \lineii{3} {Show the NFA constructed from the spec file with \file{dotty}.}
  \lineii{10}{Show the parse tree from the spec file with \file{dotty}.}
\end{tableii}

This function may raise \class{pyggy.SpecError} if there are any
errors in the spec file or \class{pyggy.InternalError}.
\end{funcdesc}



\section{\module{pyggy.pyggy} -- Grammar generation}
\declaremodule{extension}{pyggy.pyggy}
\modulesynopsis{Parser generator}
\moduleauthor{Tim Newsham}{newsham@lava.net}


This module implements the parser generator.  It is responsible for
reading in a parser spec file (in \file{.pyg} format), generating
a shift-reduce tables and emitting the tables into a
Python module.

This module can be accessed from the command line or through a call
from Python.  To run from the command line:

\begin{verbatim}
$ python pyggy.py [-d debuglevel] infile.pyg outfile.py
\end{verbatim}

\begin{funcdesc}{pyggy.pyggy.parsespec}{fname, outfname, debug=0}
The \function{parsespec} function causes the input file \code{fname}
to be processed and shift-reduce tables to be generated to
\code{fname}.  If \code{debug} is set, increasing amounts of diagnostic
output will be emitted.  The debug levels (especially higher-levels)
are subject to change but are currently:

\begin{tableii}{l|l}{exception}{Level}{Description}
  \lineii{0} {Output a count of ambiguities in the parser.}
  \lineii{1} {Output detailed diagnostics of the generated parser.}
  \lineii{2} {Turn on debugging in the parser generator engine and show precedence relations.}
  \lineii{3} {Show the LR0 state machine}
  \lineii{11}{Show the parse tree from the spec file with \file{dotty}.}
  \lineii{12}{Show the cover while showing the parse tree.}
\end{tableii}

This function may raise \class{pyggy.SpecError} if there are any
errors in the spec file or \class{pyggy.InternalError}.
\end{funcdesc}

