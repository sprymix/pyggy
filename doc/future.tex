
\chapter{Future Directions}

There's a lot left to be done.  This package is still young and
needs to be put through its paces.  There are still a number of
opportunities for improvement in the current implementation.
The grammars for the parser and lexer spec files are old and
do not yet make use of some of the newer features in PyGgy such
as the EBNF constructs.  The semantic actions of the parser are
performed in a post pass.  It might be possible to execute
these actions earlier during parsing and reduce the memory
requirements.  The internal representations used for the NFA,
DFA and shift-reduce tables may not be particularly efficient.
It is hard to know what works well and what does not without
trying it out.  Some profiling and tuning of PyGgy and PyLly
would be useful.

Beyond the obvious, there are a few things on my wish list for
making a better parsing system.  The first is to implement
scannerless parsing.  Scannerless parsing has a number of
advantages over parsing with a scanner (lexer).  PyGgy already
has many of the mechanisms that would be needed.  The second
is a mechanism for disambiguating parses during parsing.   Something
as simple as a disambiguating function could go a long way here.


\begin{seealso}
The following references were influential in the design and implementation
of this system.

\seetitle[http://www.cs.vu.nl/\%7edick/PTAPG.html]{Parsing Techniques - A Practical Guide}{A general overview of parsing technologies.}

\seetitle[http://www.cs.uu.nl/groups/ST/Visser/DisambiguationFiltersForScannerlessGeneralizedLRParsers]{Disambiguation Filters for Scannerless Generalized LR Parsers}{The precedence system used by PyGgy is based on this paper.}

\seetitle[http://citeseer.ist.psu.edu/rekers92parser.html]{Parser Generation for Interactive Environments}{The GLR parser in PyGgy is based on the parsing algorithm developed in section 1.5.1 of this paper.}

\seetitle[http://www.cs.uu.nl/groups/ST/Visser/ScannerlessGeneralizedLRParsing]{Scannerless Generalized-LR Parsing}{Building parsers without using lexers}

\seetitle[http://research.microsoft.com/research/pubs/view.aspx?msr\_tr\_id=MSR-TR-2003-32]{A Research C\# Compiler}{This paper describes a GLR-based compiler written by Microsoft Research with some novel features.  Their use of functions to disambiguate a parse is interesting.}

\seetitle[http://citeseer.ist.psu.edu/irwin01generated.html]{A Generated Parser of C++}{This paper describes a C++ parser built with a GLR parser.}

\seeurl{http://systems.cs.uchicago.edu/ply/}{PLY is another parser generator for Python.  Some examples in this text are derived from their examples.}

\seeurl{http://dparser.sourceforge.net/}{D-Parser is a GLR parser for C.  The ANSI-C example included with PyGgy is derived from a similar example for the D-Parser.}



\end{seealso}

